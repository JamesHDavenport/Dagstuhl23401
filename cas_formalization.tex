\documentclass[11pt, aspectratio=169]{beamer}
\usepackage{fontspec}

%\documentclass[11pt, handout, notes]{beamer}
\usepackage{pdfcomment}
\newcommand{\pdfnote}[1]{\marginnote{\pdfcomment[icon=note]{#1}}}
% disable comments
\renewcommand{\pdfnote}[1]{}
\usetheme{metropolis}
\metroset{block=fill}
\metroset{numbering=none}
%\metroset{numbering=fraction}
\metroset{progressbar=frametitle}


\usepackage{emoji}
\setemojifont{Apple Color Emoji}

\usepackage{tikz}
\usetikzlibrary{cd}

\tikzset{
    labl/.style={anchor=south, rotate=90, inner sep=.5mm}
}


%\makeatletter
%\setlength{\metropolis@titleseparator@linewidth}{1pt}
%\setlength{\metropolis@progressonsectionpage@linewidth}{1pt}
%\setlength{\metropolis@progressinheadfoot@linewidth}{1pt}
%\makeatother

\usepackage{numprint}
\usepackage{amsmath}
\usepackage{mathtools}
\usepackage{booktabs}
\usepackage{soul}
\usepackage[normalem]{ulem}
\usepackage{colonequals}
\usepackage{stmaryrd} % \mapsfrom

\usepackage{graphicx}
\usepackage{caption}
\usepackage{xcolor}
\definecolor{darkpurple}{rgb}{.5,.0,.5}
\definecolor{darkbrown}{rgb}{.6,.4,.2}
\usepackage{color}
\definecolor{mylinkcolor}{rgb}{0.5,0.0,0.0}
\definecolor{myurlcolor}{rgb}{0.92,0.31,0.11}
\hypersetup{pdftitle={\@title}, pdfauthor={\@author}, colorlinks, urlcolor=myurlcolor, citecolor=myurlcolor, linkcolor=mylinkcolor}
\usepackage[cmtip,all]{xy}
\usepackage{multirow}
\usepackage{wrapfig}
\usepackage[none]{hyphenat}


\DeclareMathOperator{\alg}{{al}}
\DeclareMathOperator{\Jac}{{Jac}}

\DeclarePairedDelimiter{\paren}{(}{)} % allow \paren[\big]{...}
\DeclarePairedDelimiter\abs{\lvert}{\rvert}

% Rings and spaces
\newcommand{\Q}{\mathbb{Q}}
\newcommand{\Qq}{{\Q_q}}
\newcommand{\R}{\mathbb{R}}
\newcommand{\Z}{\mathbb{Z}}
\newcommand{\C}{\mathbb{C}}
\newcommand{\N}{\mathbb{N}}
\newcommand{\Nz}{\mathbb{N}_0}
\newcommand{\Zq}{{\Z_q}}
\newcommand{\F}{\mathbb{F}}
\newcommand{\Fq}{{\F_q}}
\newcommand{\Fp}{{\F_p}}
\newcommand{\Fpbar}{ {\Fp} ^{\alg}}
\newcommand{\PP}{\mathbb{P}}
\newcommand{\Xp}{X_p}
\newcommand{\Xbar}{X^{\alg}}
\newcommand{\Xpbar}{{\Xp}^{\alg}}
\newcommand{\Ep}{E_p}
\newcommand{\Ap}{A_p}
\newcommand{\Ebar}{E^{\alg}}
\newcommand{\Epbar}{\Ep^{\alg}}
\newcommand{\Apbar}{\Ap^{\alg}}
\newcommand{\Abar}{A^{\alg}}


% operators
\newcommand{\sfloor}[1]{\lfloor {#1} \rfloor}
\newcommand{\sceil}[1]{\lceil {#1} \rceil}
\newcommand{\pxi}[2]{\frac{\partial {#1}}{\partial x_{#2}}}
\DeclareMathOperator{\frob}{Frob}
\DeclareMathOperator{\tr}{tr}
\DeclareMathOperator{\NS}{NS}
\DeclareMathOperator{\rank}{rk}
\DeclareMathOperator{\Pic}{Pic}
\DeclareMathOperator{\norm}{Nm}
\DeclareMathOperator{\prob}{Prob}
\DeclareMathOperator{\End}{End}
\DeclareMathOperator{\Hom}{Hom}
\DeclareMathOperator{\Aut}{Aut}
\DeclareMathOperator{\Gal}{Gal}
\DeclareMathOperator{\Br}{Br}
\DeclareMathOperator{\disc}{disc}
\DeclareMathOperator{\Spec}{Spec}
\DeclareMathOperator{\Km}{Kummer}

% groups
\newcommand{\AST}{\operatorname{AST}}
\newcommand{\GL}{\mathrm{GL}}
\newcommand{\GSp}{\mathrm{GSp}}
\newcommand{\Hg}{\mathrm{Hg}}
\newcommand{\Lef}{\operatorname{L}}
\newcommand{\MT}{\mathrm{MT}}
\newcommand{\M}{\mathrm{M}}
\newcommand{\PGL}{\mathrm{PGL}}
\newcommand{\SL}{\mathrm{SL}}
\newcommand{\SO}{\mathrm{SO}}
\newcommand{\GO}{\mathrm{GO}}
\newcommand{\Sp}{\mathrm{Sp}}
\newcommand{\SU}{\mathrm{SU}}
\newcommand{\U}{\mathrm{U}}
\newcommand{\USp}{\mathrm{USp}}
\newcommand{\ST}{\mathrm{ST}}
\newcommand{\TL}{\operatorname{TL}}
\newcommand{\Or}{\mathrm{O}}

\newcommand{\magic}{\emoji{magic-wand} \emoji{top-hat} }
\newcommand{\doublemagic}{\emoji{magic-wand} \emoji{zap} \emoji{top-hat}  \emoji{collision}}


\title{A Formalizer, a Mathematician, and a Computer Algebra System Walk into a Bar: Bridging Formal and Computational Mathematics}

\date{November 4, 2023, Simons Collaboration Meeting\\

Slides available at \url{edgarcosta.org}\\
 Joint work with Alex J. Best, Mario Carneiro, and James Davenport.}
\author{Edgar Costa (MIT)}



\begin{document}


\begin{frame}
  \titlepage
\end{frame}

\begin{frame}[fragile]{What is true?}
    \begin{block}{Question}
        Do I believe the output from a computer algebra system?
    \end{block}

\pause
    \begin{theorem}
        The number $3 \cdot 2^{189} + 1$ is a prime number.
    \end{theorem}
    \begin{block}{Proof \magic}
        \vspace{-0.5em}
    \begin{semiverbatim}
sage: (3 * 2^189 + 1).is_prime(proof=True)
True
magma > IsPrime(3 * 2^189 + 1 : Proof:=true);
true
gp ? isprime(3 * 2^189 + 1)
%1 = 1
\end{semiverbatim}
\end{block}
\end{frame}

\begin{frame}[fragile]{What is true?}
    \begin{theorem}
        The number $3 \cdot 2^{189} + 1$ is a prime number.
    \end{theorem}
    \begin{block}{Proof \doublemagic }
        Take $n \colonequals 3 \cdot 2^{189} + 1$.
        It is sufficient to exhibit $a$ such that
        \vspace{-0.5em}
        $$
        1 \notin \{a^{(n-1)/2} \bmod{n},  a^{(n-1)/3} \bmod{n} \}.
        $$
        \vspace{-3em}
\begin{semiverbatim}
sage: n = 3 * 2^189 + 1
....: a = Zmod(n)(10)
....: 1 not in [a^((n-1)/2), a^((n-1)/3)]
True
\end{semiverbatim}
\vspace{-1em}
\end{block}
Since $n$ is a proth number, it is enough exhibit $a$ such that $a^{(n-1)/2} \equiv -1 \bmod n$.

There several other possible prime certificates.

\end{frame}

\begin{frame}[fragile]{What is true?}
    \begin{block}{Theorem}
        The class group of $K \colonequals \Q(\sqrt{5}, \sqrt{-231}) = \href{https://www.lmfdb.org/NumberField/4.0.1334025.9}{\texttt{4.0.1334025.9}}$ is $C_{2}\times C_{2}\times C_{12}$.
    \end{block}
    \vspace{-0.2em}
    \begin{block}{Proof \magic}
        \vspace{-1em}
\begin{semiverbatim}
sage: K.class_group().invariants()
(12, 2, 2)
magma> Invariants(ClassGroup(K));
[ 2, 2, 12 ]
julia> class_group(K)[1]
GrpAb: (Z/2)^2 x Z/12
\end{semiverbatim}
        \vspace{-1em}
    \end{block}
    \vspace{-0.2em}
\begin{block}{Proof \doublemagic}
    \texttt{magma> Degree(HilbertClassField(K));}
    \\
    \texttt{48}\\
    \emoji{skull-and-crossbones} \texttt{segmentation fault (core dumped)}
    \end{block}
\end{frame}

\begin{frame}{What is true?}
\vspace{-0.9em}
    \begin{small}
    \begin{align*}
        C_1 &\colon
        y^2 + (x + 1)y = x^5 + 23x^4 - 48x^3 + 85x^2 - 69x + 45
        \\
        C_2 &\colon
        y^2 + x y = -x^5 + 2573 x^4 + 92187 x^3 + 2161654285 x^2 + 406259311249 x + 93951289752862
    \end{align*}
\end{small}
\vspace{-1em}
    \begin{block}{Theorem}
        There is an isogeny of degree $31^2$ between $\Jac(C_1)$ and $\Jac(C_2)$.
    \end{block}
    \begin{block}{Proof \magic \quad 3h}
        Compute the isogeny class via Bommel--Chidambaram--Costa--Kieffer:\\
        \texttt{sage -python genus2isogenies.py ...}
    \end{block}
    \begin{block}{Proof \doublemagic \quad 6.5h}
    Produce a divisor in $C_1 \times C_2$ via Costa--Mascot--Sijsling--Voight:\\
    \texttt{magma> Correspondence(C1, C2, heuristic\_isogeny);}
    \\
    \texttt{...}
\end{block}
\end{frame}


\begin{frame}{Spectrum of options}
    \begin{itemize}
            \item \emoji{scroll} Generate certificates of correctness a posteriori
        \begin{itemize}
            \item Primality proving
            \item Homomorphisms between Jacobians
            \item LLL lattice basis reduction
        \end{itemize}
        \pause
        \item \emoji{gear} Formalize the algorithm
            \begin{itemize}
                \item Smith and Hermite normal form
                \item Factorisation over $\Z[x]$
                \item LLL lattice basis reduction algorithm
                \item Tate's algorithm
            \end{itemize}
        \pause
        \item \emoji{magic-wand} By pure thought generate an alternative proof
            \pause
            \begin{itemize}
                \item a magician never reveals their secrets
            \end{itemize}
    \end{itemize}
\end{frame}

\begin{frame}{\emoji{gear} Formalization of factorization over $\Z[x]$}
\vspace{-0.5em}
\begin{center}
\includegraphics[width=0.8\textwidth]{images/Zassenhaus_graph.pdf}\\
\end{center}
\vspace{-0.5em}
by Divasón--Joosten--Thiemann--Yamada
\end{frame}

\begin{frame}{\emoji{gear} LLL lattice basis reduction algorithm}
\vspace{-0.5em}
    \begin{center}
\includegraphics[height=0.85\textheight]{images/LLL_graph.pdf}\\
\end{center}
by Thiemann--Bottesch--Divasón--Haslbeck--Joosten--Yamada
\end{frame}

\begin{frame}{\emoji{gear} Tate's algorithm (work in progress by Best--Dahmen--Huriot-Tattegrain)}
    \begin{itemize}
        \item Some of the output is out of reach to be formalized:
            \begin{itemize}
                \item Kodaira symbol
                \item Conductor exponent
                \item Tamagawa number
                \item ...
            \end{itemize}
        \item They verified that the algorithm terminates under some mild assumptions
        \item Works in characteristic 2 and 3
        \item Verified output for some explicit families, e.g., $y^2 = x^3 + p$ gives $I_1$ for $p >5$
        \item Verified the local data on LMFDB ($\sim\!13$ million curves) in $\sim\!10$ minutes
        \item Future: show that the output is invariant under change of coordinates
    \end{itemize}
\end{frame}

\begin{frame}{The sweet spot}
    \begin{itemize}
        \item \emoji{honey-pot} Generate bread crumbs for a certificate along the way
            \begin{itemize}
                \item Primality testing via elliptic curves
                \item Factorisation over $\Z[x]$
                \item Class group computation?
            \end{itemize}
        \item \emoji{scroll} Generate certificates of correctness a posteriori
            \begin{itemize}
                \item Primality proving
                \item Homomorphisms between Jacobians
                \item LLL lattice basis reduction algorithm
            \end{itemize}
        \item \emoji{gear} Formalize the algorithm
            \begin{itemize}
                \item Smith and Hermite normal form
                \item Factorisation over $\Z[x]$
                \item LLL lattice basis reduction algorithm
                \item Tate's algorithm
            \end{itemize}
        \item \emoji{magic-wand} By pure thought generate an alternative proof
            \begin{itemize}
                \item a magician never reveals their secrets
            \end{itemize}
    \end{itemize}
\end{frame}
\begin{frame}[fragile]{\emoji{honey-pot} Factorisation over $\Z[x]$ (Best--Carneiro--Costa--Davenport)}
    \only<-2>{
    \begin{theorem}[Mignotte]
        Take $f, g \in \Z[X]$, and let $n = \deg f$.
        If $g$ divides $f$, then
        $$ \|g\|_{\infty} \le \binom{n-1}{\lceil n/2 \rceil}(\|f\|_2 + lc(f)) \equalscolon B_f$$
    \end{theorem}
}
    \pause

    \only<-4>{
    To show that $f$ is irreducible, is enough to give a factorization of $f$ over $\Z/p^e [x]$, with $p^e > 2B_f + 1$, such that no nontrivial factor lifts as a factor of $f$ over $\Z[x]$.

    Such factorization is free! Already part of the factorization algorithm.
}

    \pause
    \begin{theorem}
        $f \colonequals x^{6} - 3 x^{5} + 5 x^{4} - 5 x^{3} + 5 x^{2} - 3 x + 1$ is irreducible
    \end{theorem}
    \begin{block}{Proof \emoji{honey-pot}}
        \begin{itemize}
            \item[\emoji{framed-picture}]
\texttt{sage: f.is\_irreducible()\\
True}
\pause
\item[\emoji{mag-right}]
\begin{itemize}
    \item[\emoji{gift}] Over $\Z/3^e[x]$ $f$ factors as $g \cdot h$, with $\deg g = \deg h = 3$
    \item[\emoji{gear}] the putative lifts to $\Z[x]$ do not divide $f$
\end{itemize}
        \end{itemize}
    \end{block}
    \pause
    Our goal is to build a \emph{tactic} in lean to automatically generate such formal proofs.

    \pause
    Can we do a similar thing for class group computations? \emoji{teddy-bear}
\end{frame}

\end{document}

